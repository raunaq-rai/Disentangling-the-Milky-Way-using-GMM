\documentclass[a4paper,10pt]{article}

\usepackage[a4paper,margin=1in]{geometry}
\usepackage{multicol}       % For multiple columns
\usepackage{graphicx}       % For including images if needed
\usepackage{parskip}        % No paragraph indent, adds space between
\usepackage{titlesec}       % For section title formatting
\usepackage{mathptmx}
\usepackage{amsmath,amssymb}
\usepackage[authoryear]{natbib}
\usepackage[colorlinks=true, citecolor=blue, linkcolor=black, urlcolor=blue]{hyperref}


% Optional: Customize section formatting
\titleformat{\section}{\large\bfseries}{\thesection}{1em}{}
\titleformat{\subsection}{\normalsize\bfseries}{\thesubsection}{1em}{}

\begin{document}

\begin{center}
    {\Large \textsc{Disentangling the Components of the Milky Way}}\\[0.2cm]
    {\textsc{Inferring the Structure of the Milky Way in Phase-Space Using Gaussian Mixture Modelling with Extreme Deconvolution}}\\[0.2cm]
    Raunaq Singh Rai \quad | \quad MPhil Data Intensive Science \quad | \quad University of Cambridge
\end{center}

\begin{multicols}{2}
% Start of two-column content

\subsection*{Motivation and Scientific Justification}

Understanding how the Milky Way assembled its structural components is central to the field of 
Galactic Archaeology. Observational advances, especially through the \textit{Gaia} mission, allow detailed investigation 
of the chemical and kinematic signatures that trace the Galaxy’s formation history. Among these, the existence of a 
very-metal-poor (VMP) stellar disc presents a key test of disc formation timescales.

Standard models posit that the Galactic disc formed relatively late, from gas that had already been enriched by several 
generations of star formation. However, if stars with disc-like orbits are observed at metallicities below $[\mathrm{Fe/H}] 
\lesssim -1.5$, it would suggest that the disc began assembling much earlier—potentially contemporaneous with or even before 
major accretion events such as Gaia–Sausage/Enceladus. This would challenge prevailing hierarchical-formation narratives and 
place new constraints on the interplay between in-situ and ex-situ star formation.

The recent study by Zhang et al.~\cite{zhang2024existencemetalpoordiscmilky} introduced a compelling statistical framework for 
testing this idea using Extreme-Deconvolution Gaussian Mixture Modelling (XD-GMM). Their findings argue against a cold, 
rotationally supported VMP disc. However, the method’s sensitivity to substructure, selection effects, and chemical splits 
(such as alpha abundance) invites closer scrutiny.

This project builds on their approach in two ways. First, we reproduce their metallicity-binned XD-GMM decomposition using 
the same Gaia DR3-based red giant catalogue. Second, we extend their analysis by splitting the sample into high- and low-alpha 
sequences using methodology proposed by Viswanathan et al. \cite{Vis2024}, testing whether disc-like kinematics emerge at different metallicities within each branch. In doing so, we aim to 
clarify whether previous disc-like signals at low metallicity reflect genuine early disc formation or instead arise from noise, 
misclassification, or accreted debris.

The scientific justification rests on the idea that even a small population of metal-poor stars with coherent, 
disc-like kinematics could alter our understanding of the Galaxy’s early history. By carefully quantifying kinematic 
structure across chemical space—and explicitly accounting for measurement uncertainties—we contribute to a clearer, more 
robust reconstruction of the Milky Way’s formation pathway.


\bibliographystyle{unsrt}  % Or use apalike, mnras, etc., depending on your needs
\bibliography{references}

% End of two-column content
\end{multicols}



\end{document}
